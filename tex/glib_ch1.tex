\chapter{What is GLib?}

\begin{itemize}
\item What is GLib?
\item Where did GLib come from?
\item How to compile and link against GLib.
\end{itemize}

\section{What is GLib?}

GLib is a library that essentially turns C into a higher-level
language.  Many such libraries have been written, but GLib is popular
because it is the foundation for GTK+ and the various desktop
environments.  It has turned from being a small collection of
utilities useful for C programming, to being a library that you can't
live without if you program in C.

\section{Where did GLib come from?}

In 1996, the first versions of the GNU Image Manipulation Program (the
GIMP) used Motif, a proprietary widget toolkit for Unix.  The GIMP's
authors understood that this was an inconvenient situation, because it
meant that although the GIMP itself was released under a free software
license, the GPL, people could in effect not recompile the GIMP if
they didn't have a Motif license.

At that time, there were a few free widget toolkits available for
Unix, but none seemed to be complete enough for the GIMP and its heavy
graphical requirements.  So, the GIMP's authors took a break to write
their own widget toolkit, and then rewrite the GIMP to use it.

This toolkit was GTK, the GIMP Tool Kit.  Later, GTK was updated to
have a full object system for C and other utilities, and it was
renamed to GTK+.  Apart from the object system and the widget toolkit,
it included GLib, a small library of C functions and macros to make
C programming easier and more portable.  GLib contained things like
linked lists and hash tables, resizable arrays and strings, and other
things that almost every program would find convenient.

Initially, GTK+ was shipped as part of the GIMP's source code
distribution.  People who wanted to write programs with GTK+ first had
to install the GIMP.  Around the time the Gnome project started, GTK+
got separated from the GIMP and started to be shipped as a stand-alone
library that contained GLib as part of itself.

GTK+ 2.0 in turn got split into two separate libraries:  GLib, for the
non-graphical part and the object system, and GTK+ proper, the widget
toolkit.  Over time, GLib has added various things:  a threading
abstraction, an introspectible object and type system, loadable
modules, and GIO, a full I/O library with modern features like
asynchronous operations.

GLib has grown organically, in a well-controlled way.  Many of its
features have been ``pushed down in the stack'' of libraries from
higher-level places, as those bits of code have matured into
general-purpose utilities.  For example, GIO embodies the experience
gained from writing first mc-vfs, the Midnight Commander file
manager's virtual file system abstraction, then gnome-vfs for the
Nautilus file manager, and finally thinking about how synchronous and
asynchronous I/O operations need to happen in a convenient way for the
particular needs of a multithreaded graphical program.

\section{How to compile and link against GLib}

Since GLib uses \verb|pkg-config|, compiling programs that use it is
easy.  For example,

\begin{code}{}
\$ ls
main.c
\$ gcc -o myprogram main.c \$(pkg-config --cflags --libs glib-2.0)
\$ ./myprogram
Hello, GLib!
\$
\end{code}

For pkg-config's purposes, GLib's sub-libraries are as follows:

\begin{itemize}
\item \verb|glib-2.0| -- The low-level GLib library, with data
  structures and utilities.
\item \verb|gmodule-2.0| -- GModule, a system of loadable modules for
  applications that require plug-in type functionality.
\item \verb|gobject-2.0| -- GObject, the type and object system in GLib.
\item \verb|gio-2.0| -- GIO, an I/O library for file and network
  operations with many features.
\item \verb|gthread-2.0| -- Threads and thread pools.
\end{itemize}
